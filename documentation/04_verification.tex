\chapter{Tests}
This chapter shows what we have done in terms of software verification. We performed all our tests using two different methods: via the UI and using an automated test tool from a terminal, by means of predefined test cases.


\section{Controller Tests}
\label{subsec:controllertests}

Verification of controllers is the most important part, since all the logics and security chekcs are implemented at this level. We decided to take special care of controller testing and defined an easy and complete way to test all the controller systematically.

Our tests take into account the way we implemented all the controller tasks, so they can be considered \emph{white-box testing}. 

\subsection{CodeIgniter}
As we already said previously, we used a well-known MVC framework named \emph{Code Igniter}. It provides a set of functionalities that we did not need to test. For example, the URL mapping into controller functions, or the PHP database interactions, or even more sophisticated tools to parse template files, to perform form input checks and so forth. This framework has been tested by its developers, so we simply rely on it. It has also been tested in terms of speed, so that we don't worry too much about this kind of performance. All we need to test is our code, that is, all the controllers, models and views.

\paragraph{URL mapping}
In Code Igniter every URL is mapped to a single function, inside a single controller class. This creates a 1-1 correspondence between URL's and controller functions. As Code Igniter has been implemented, a URL that does not correspond to any function, will generate an error.

For this reason, we can talk about URL's instead of controller functions, since they are (almost) the same thing. Our test sets exploit this characteristic.

\paragraph{Views and Models}
All the views are simple HTML code and we believe that it is not really crucial. All the models rely on Code Igniter's SQL wrappers, so they don't really need to be tested. All the logics are inside the controllers, therefore we only focused on them. Despite that, we describe our template implementation of models, since we believe that it is part of testing.

\paragraph{Model result handling}
Every controller and every model function has been implemented using the same pattern. All model functions return a specific value to tell the controller what exactly happened during their execution (query error? other error? data inserted/deleted/updated correctly?, etc.). Hence, a generic controller/model execution will be:
\begin{enumerate}
\item ULR gets mapped by Code Igniter and a specific controller function gets executed
\item Some operations (checks, etc.)
\item Call to a model function. Store the result
\item Check the result and change the UI based on that
\item Go again to 2) or exit
\end{enumerate}

\subsection{Our Test Framework}
All our test exploit the previous schema and a special function that we designed. The class ``Auth'', in fact, implements a method called \emph{check} that takes an array of checks to be performed, performs them and if even one of them fails, it stops its execution and returns. Otherwise, if all the checks are passed, it returns TRUE. So, the previous schema can be described in more details:
\begin{enumerate}
\item ULR gets mapped by Code Igniter and a specific controller function gets executed
\item Call to function \emph{check}, passing an array of conditions to be checked
\item If any of them fails, stop execution providing useful messages via the UI
\item Calls to models, UI or else, as before.
\end{enumerate}

\paragraph{Test cases based on the check function}
An example of a call to \emph{check} is the following:
\begin{verbatim}
$check = $this->auth->check(array(
    auth::CurrLOG,
    auth::CurrPAT,
    auth::HCP, $hdp_id
));
if ($check !=== TRUE) return;
\end{verbatim}

In this example, we are forcing the current user to be logged in, to be a patient and we are checking whether the variable \$hcp\_id refers to an healthcare provider. If any of them fails, we stop the current execution. Note that the function \emph{check} will automatically set the UI to display an appropriate error message.

This gives us a very simple way of implementing our test sets. For the previous case we simply need to execute:
\begin{enumerate}
\item A test being a not-logged-in user
\item A test not being a patient
\item A test passing an id that does not refer to an healthcare provider
\item A test in which all of the checks would be successfully passed
\end{enumerate}
We would expect a success message only for the last test. Whereas, we expect an error in all the other cases. Now we address the problem of making automated tests.

\paragraph{Tests via the UI}
We believe that using the UI, which means using the browser interface, is a very important way to test a web application such ours. We performed all possible paths that a user could go through, using our website. Sometimes, when possible, we forces some path using the location bar directly, which means putting a specific URL in the browser, instead of using links in the web page. We based our test cases on what we discussed previously and also on our knowledge of all the requirements (in terms of expected behaviours/results).

\paragraph{Automated tests}
We also wanted to create a framework to make automated tests possible. We realized that the only way to test unauthorized paths and requests was hard-coding URL's. An automated test framework can help us runs tens of test cases all at once. It is also a good tool for debugging, since it helps us to identify unexpected results easily.

We based our design on the Unix function \emph{curl}, that can be used to make an http request to a server and get the response back from it. It is a very complete tool, in the sense that it also provides the use of POST variables, cookies, SSL connections and so forth.

To make everything possible, we needed to define a special ``test mode'' for the UI. Activating the test mode, the UI does not behave as usual. Instead, it always answers with a special ``result string''. These are all the possible results:
\begin{description}
\item CI\_ERROR In case Code Igniter produces an error
\item CI\_PHP\_ERROR In case a PHP error occurred
\item CI\_404\_ERROR In case the URL does not correspond to any controller function
\item QUERY\_ERROR In case a query error occurred
\item CTR\_ERROR In case the controller set an error message to the UI
\item REDIRECTED In case the controller set the UI to redirect the current page
\item OK\_MESSAGE In case the controller set the UI to display a message (not error)
\item ALL\_OK In case the controller terminates without any error and any special message
\end{description}

Based on these result messages and on the \emph{check} function calls, we defined a set of test cases for each controller function. We wrote and used a C++ program to execute them via \emph{curl}, one at a time. This program reads a file of test cases and executes them in order. If one of them fails, i.e. if a result does not match the expected value, it returns, showing the test case that caused the error.

This is an example of a file with test cases that this program can read\footnote{This file can be found in /documentation/tests/TEST\_SET\_1}:
\begin{verbatim}
# Test Home controller (for these you don't need to be logged in)
/home						ALL_OK
/home/sitemap				ALL_OK
/home/retrieve_password		ALL_OK
/home/logout				REDIRECTED

# These are errors, since you need to be logged in
/profile					CTR_ERROR
/home/login					CTR_ERROR

# Now, try to login in and save the cookie in a local file
/home/login					REDIRECTED		| -F "email=patient1@yahoo.com" -F "password=cameraman" -c cookie

# Then, use this cookie to access restricted areas of the website
/profile					ALL_OK			| -b cookie -c cookie

# Invalid login
/home/login					CTR_ERROR		| -F "email=patient1@yahoo.com" -F "password=camera" -c cookie

# Other fuzz tests
sjgjfsgbfgb					CI_404_ERROR
/home/nothing				CI_404_ERROR
/hello						CI_404_ERROR
\end{verbatim}
As you can see, it is also possible to use cookies and POST variables. We believe that this tool is very powerful and combined with the way we implemented all the controllers, it provides a good verification of our entire project, via testing.

\paragraph{Test files}
Here we show our test files for connections, groups and referrals.

\begin{verbatim}
########################################################################
###################   TEST FILE FOR CONNECTIONS   ######################
########################################################################
#                                                                      #
# Use this syntax:                                                     #
#   <url>   <expected_result_string>   [ "|" <curl_extra_options> ]    #
#                                                                      #
# Examples:                                                            #
#   /home/sitemap    ALL_OK                                            #
#   /profile         ALL_OK      | -b cookie_file                      #
#                                                                      #
########################################################################

# Log in as patient 1
/home/login					REDIRECTED		| -F "email=patient1@email.com" -F "password=pat1" -c cookie_pat_1
# Log in as patient 2
/home/login					REDIRECTED		| -F "email=patient2@email.com" -F "password=pat2" -c cookie_pat_2
# Log in as patient 3
/home/login					REDIRECTED		| -F "email=patient3@email.com" -F "password=pat3" -c cookie_pat_3
# Log in as patient 4
/home/login					REDIRECTED		| -F "email=patient4@email.com" -F "password=pat4" -c cookie_pat_4
# Log in as patient 5
/home/login					REDIRECTED		| -F "email=patient5@email.com" -F "password=pat5" -c cookie_pat_5
# Log in as patient 10
/home/login					REDIRECTED		| -F "email=patient10@email.com" -F "password=pat10" -c cookie_pat_10
# Log in as hcp 51
/home/login					REDIRECTED		| -F "email=doctor51@email.com" -F "password=doc51" -c cookie_doc_51
# Log in as hcp 52
/home/login					REDIRECTED		| -F "email=doctor52@email.com" -F "password=doc52" -c cookie_doc_52
# Log in as hcp 53
/home/login					REDIRECTED		| -F "email=doctor53@email.com" -F "password=doc53" -c cookie_doc_53
# Log in as hcp 54
/home/login					REDIRECTED		| -F "email=doctor54@email.com" -F "password=doc54" -c cookie_doc_54
# Log in as hcp 55
/home/login					REDIRECTED		| -F "email=doctor55@email.com" -F "password=doc55" -c cookie_doc_55
# Log in as hcp 60
/home/login					REDIRECTED		| -F "email=doctor60@email.com" -F "password=doc60" -c cookie_doc_60

################ VIEW CONNECTED AND UNCONNECTED PROFILES ###############
# Patient 1 is connected with pats 2, 3, 4 and 5
/profile/user/2				ALL_OK			| -b cookie_pat_1
/profile/user/1				ALL_OK			| -b cookie_pat_2

/profile/user/4				ALL_OK			| -b cookie_pat_1
/profile/user/1				ALL_OK			| -b cookie_pat_4

# Patient 1 is connected with docs 51, 52, 53, 54
/profile/user/51			ALL_OK			| -b cookie_pat_1
/profile/user/1				ALL_OK			| -b cookie_doc_51
/profile/user/52			ALL_OK			| -b cookie_pat_1
/profile/user/1				ALL_OK			| -b cookie_doc_52
/profile/user/53			ALL_OK			| -b cookie_pat_1
/profile/user/1				ALL_OK			| -b cookie_doc_53
/profile/user/54			ALL_OK			| -b cookie_pat_1
/profile/user/1				ALL_OK			| -b cookie_doc_54
/profile/user/55			ALL_OK			| -b cookie_pat_1
/profile/user/1				ALL_OK			| -b cookie_doc_55

# Try to see not connected profiles
/profile/user/6				ALL_OK			| -b cookie_pat_1
/profile/user/7				ALL_OK			| -b cookie_pat_1
/profile/user/8				ALL_OK			| -b cookie_pat_1
/profile/user/9				ALL_OK			| -b cookie_pat_1
/profile/user/10			ALL_OK			| -b cookie_pat_1
/profile/user/56			ALL_OK			| -b cookie_pat_1
/profile/user/57			ALL_OK			| -b cookie_pat_1
/profile/user/58			ALL_OK			| -b cookie_pat_1
/profile/user/59			ALL_OK			| -b cookie_pat_1
/profile/user/60			ALL_OK			| -b cookie_pat_1

# Try to change account type to private of pat 6
/settings/change_privacy_do	OK_MESSAGE		| -F "level=1" -b cookie_pat_10

# Try to see his profile
/profile/user/10			CTR_ERROR		| -b cookie_pat_1
/profile/user/10			CTR_ERROR		| -b cookie_hcp_51

# Restore previous privacy for pat 10
/settings/change_privacy_do	OK_MESSAGE		| -F "level=0" -b cookie_pat_10

################### CREATE NEW CONNECTIONS #############################
# Patient 1 asks for a new connection to hcp 60
/connections/request/60		OK_MESSAGE		| -b cookie_pat_1
# Then hcp 60 accets it
/connections/accept/1		ALL_OK			| -b cookie_doc_60

# Patient 1 asks for a new connection to patient 10
/connections/request/10		OK_MESSAGE		| -b cookie_pat_1
# Then patient 10 accets it
/connections/accept/1		ALL_OK		| -b cookie_pat_10

# Patient 1 asks the same connections again
/connections/request/60		CTR_ERROR		| -b cookie_pat_1
/connections/request/10		CTR_ERROR		| -b cookie_pat_1

# And the others accept again
/connections/accept/1		CTR_ERROR		| -b cookie_doc_60
/connections/accept/1		CTR_ERROR		| -b cookie_pat_10

# Now a doctor tries to request a connection to another doctor
/connections/request/60		OK_MESSAGE		| -b cookie_doc_51

# And then, to a patient
/connections/request/10		CTR_ERROR		| -b cookie_doc_51

# Now, two accounts try to accept connection not for them
/connections/accept/51		CTR_ERROR		| -b cookie_doc_52
/connections/accept/51		CTR_ERROR		| -b cookie_pat_1

# And finally 60 accepts his request
/connections/accept/51		ALL_OK			| -b cookie_doc_60

# Try to cancel a connection already accepted
/connections/cancel/60		ALL_OK			| -b cookie_doc_51

# Try to reject a connection already accepted
/connections/reject/51		CTR_ERROR		| -b cookie_doc_60

# Try to legally destroy a connection
/connections/destroy/51		ALL_OK			| -b cookie_doc_60

# Now the other tries to destroy it again
/connections/destroy/60		CTR_ERROR		| -b cookie_doc_51

# Restore previous privacy for pat 10
/settings/change_privacy_do	OK_MESSAGE		| -F "level=0" -b cookie_pat_10

# Try to request a connection to pat 6 who is private
/connections/request/10		CTR_ERROR		| -b cookie_pat_1

# Restore previous privacy for pat 10
/settings/change_privacy_do	OK_MESSAGE		| -F "level=0" -b cookie_pat_10
\end{verbatim}

\begin{verbatim}
########################################################################
#####################   TEST FILE FOR GROUPS   #########################
########################################################################
#                                                                      #
# Use this syntax:                                                     #
#   <url>   <expected_result_string>   [ "|" <curl_extra_options> ]    #
#                                                                      #
# Examples:                                                            #
#   /home/sitemap    ALL_OK                                            #
#   /profile         ALL_OK      | -b cookie_file                      #
#                                                                      #
########################################################################

#Groups Controller

# Now, try to login in and save the cookie in a local file

#Patient 1
/home/login					REDIRECTED		| -F "email=patient1@email.com" -F "password=pat1" -c cookie_p1

#Patient 2,3 connected with Patient 1
/home/login					REDIRECTED		| -F "email=patient2@email.com" -F "password=pat2" -c cookie_p2
/home/login					REDIRECTED		| -F "email=patient3@email.com" -F "password=pat3" -c cookie_p3

#Patient 10 not connected with Patient 1
/home/login					REDIRECTED		| -F "email=patient10@email.com" -F "password=pat10" -c cookie_p10

#HCP 51,52 connected with Patient 1
/home/login					REDIRECTED		| -F "email=doctor51@email.com" -F "password=doc51" -c cookie_h51
/home/login					REDIRECTED		| -F "email=doctor52@email.com" -F "password=doc52" -c cookie_h52

#HCP 56 not connected with Patient 1
/home/login					REDIRECTED		| -F "email=doctor56@email.com" -F "password=doc56" -c cookie_h56

# Invalid login
# wrong password
/home/login					CTR_ERROR		| -F "email=patient1@email.com" -F "password=camera" -c cookie_p1_2
# wrong email
/home/login					CTR_ERROR		| -F "email=patient1@yahoo.com" -F "password=pat1" -c cookie_p1_3

# Errors if not logged in
/groups						CTR_ERROR
/groups/lists				CTR_ERROR
/groups/lists/all			CTR_ERROR
/groups/lists/mine			CTR_ERROR
/groups/lists/invites		CTR_ERROR
/groups/create				CTR_ERROR
/groups/delete				CTR_ERROR
/groups/edit				CTR_ERROR
/groups/members				CTR_ERROR
/groups/members/list		CTR_ERROR
/groups/members/edit		CTR_ERROR
/groups/members/join		CTR_ERROR
/groups/members/leave		CTR_ERROR
/groups/members/delete		CTR_ERROR
/groups/members/invite		CTR_ERROR


# if logged in 
/groups						ALL_OK			| -b cookie_p1 -c cookie_p1
/groups/lists				ALL_OK			| -b cookie_p1 -c cookie_p1
/groups/lists/all			ALL_OK			| -b cookie_p1 -c cookie_p1
/groups/lists/mine			ALL_OK			| -b cookie_p1 -c cookie_p1
/groups/lists/invites		ALL_OK			| -b cookie_p1 -c cookie_p1

######################## GROUP MEMBERS ########################

# Fails. expects a fn and a group id. 
/groups/members				CTR_ERROR		| -b cookie_p1 -c cookie_p1 

### LISTING GROUP MEMBERS ###
# Listing group members of a group you are in
/groups/members/list/40		ALL_OK			| -b cookie_p1 -c cookie_p1

# Listing group members of a group you are not in
/groups/members/list/43		CTR_ERROR		| -b cookie_p1 -c cookie_p1

### EDITING GROUP MEMBERS ###
# edit a member of a group you are in, given you have permissions
/groups/members/edit_do/40/2		ALL_OK		| -b cookie_p1 -c cookie_p1 -F "permissions=1"

# edit a member of a group you are in, when you dont have permissions 
/groups/members/edit_do/41/2		CTR_ERROR	| -b cookie_p10 -c cookie_p10 -F "permissions=1"

# Try to edit a person who is not in the group
/groups/members/edit_do/40/63		CTR_ERROR	| -b cookie_p2 -c cookie_p2 -F "permissions=2"


### JOINING A GROUP ###

# Correct input
# Try joining a patient only public group as a patient
/groups/members/join/20			ALL_OK		| -b cookie_p1 -c cookie_p1

# Try joining a hcp only public group as a hcp
/groups/members/join/37			ALL_OK		| -b cookie_h51 -c cookie_h51

# Try joining a mixed public group as a patient 
/groups/members/join/38			ALL_OK		| -b cookie_p1 -c cookie_p1

# Try joining a mixed public group as a hcp
/groups/members/join/38			ALL_OK		| -b cookie_h51 -c cookie_h51

# Incorrect input
# Try joining a group you already are a member of
/groups/members/join/1			CTR_ERROR	| -b cookie_p1 -c cookie_p1

# Try joining a patient only public group as an hcp 
/groups/members/join/1			CTR_ERROR	| -b cookie_h51 -c cookie_h51

# Try joining a hcp only public group as an patient 
/groups/members/join/21			CTR_ERROR	| -b cookie_p1 -c cookie_p1

# Try joining a group that doesn't exist
/groups/members/join/1089		CTR_ERROR	| -b cookie_p1 -c cookie_p1

### LEAVING A GROUP ###

# leave a group you are in 
/groups/members/leave/2			ALL_OK		| -b cookie_p1 -c cookie_p1

# leave a group you are a creator of
/groups/members/leave/1			CTR_ERROR	| -b cookie_p1 -c cookie_p1

# leave a group  you are not in
/groups/members/leave/21 		CTR_ERROR	| -b cookie_p1 -c cookie_p1

# leave a group that does not exist 
/groups/members/leave/1089		CTR_ERROR	| -b cookie_p1 -c cookie_p1

### DELETING A GROUP ###

# delete a group i am in and i have permission to delete
/groups/delete/1			ALL_OK		| -b cookie_p3 -c cookie_p3

# delete a group i am in and i dont have permission to delete
/groups/delete/1			CTR_ERROR	| -b cookie_p10 -c cookie_p10

# delete a group that i am not in
/groups/delete/43			CTR_ERROR		| -b cookie_p1 -c cookie_p1

# delete a group that does not exist 
/groups/delete/9890			CTR_ERROR		| -b cookie_p1 -c cookie_p1

### CREATING A GROUP ###

# create a group with all information
/groups/create_do/ 			ALL_OK			| -b cookie_p1 -c cookie_p1 
                                              -F "name=TestGroup" 
                                              -F "description=TestDescrip"
                                              -F "public_private=0" -F "group_type=2"

# create a patient only group with an hcp account
/groups/create_do/ 			CTR_ERROR		| -b cookie_h51 -c cookie_h51 
                                              -F "name=TestGroup2" 
                                              -F "description=TestDescrip2" 
                                              -F "public_private=0" -F "group_type=0"

# create a hcp only group with an patient account
/groups/create_do/ 			CTR_ERROR		| -b cookie_p1 -c cookie_p1 
                                              -F "name=TestGroup3" 
                                              -F "description=TestDescrip3" 
                                              -F "public_private=0" -F "group_type=1"

# create a group with no information
/groups/create_do/ 			CTR_ERROR		| -b cookie_p1 -c cookie_p1 
                                              -F "name=" -F "description=" 
                                              -F "public_private=" -F "group_type="

# create a group with some information
/groups/create_do/ 			CTR_ERROR		| -b cookie_p1 -c cookie_p1 
                                              -F "name=" -F "description=" 
                                              -F "public_private=0" -F "group_type=1"

### EDIT A GROUP ###
# edit group i am a member of and have permission with all info
/groups/edit_do/2			ALL_OK		| -b cookie_p2 -c cookie_p2 
                                              -F "name=NameChange" -F "description=DescripChange"

# edit group i am a member of and have permission with no posts
/groups/edit_do/2			CTR_ERROR	| -b cookie_p2 -c cookie_p2 
                                              -F "name=" -F "description="

# edit group i am a member of but dont have permission
/groups/edit_do/1			CTR_ERROR	| -b cookie_p10 -c cookie_p10 
                                              -F "name=NameChange2" -F "description=DescripChange2"

# edit group i am not a member of
/groups/edit_do/43			CTR_ERROR	| -b cookie_p1 -c cookie_p1 
                                              -F "name=NameChange3" -F "description=DescripChange3"

### DELETE A GROUP MEMBER ###

# delete a member of the group, given you have permission
/groups/members/delete/2/9		ALL_OK		| -b cookie_p2 -c cookie_p2

# delete a member of the group, if you dont have permission
/groups/members/delete/1/9		CTR_ERROR		| -b cookie_p10 -c cookie_p10

# delete a member of a group that you are not in
/groups/members/delete/20/9		CTR_ERROR		| -b cookie_p1 -c cookie_p1


### INVITATIONS TO A GROUP ###
## NOTE: curl seems to not support array form posts ... ids is an array of ids..."##
# Inviting a connected friend to a private group that I am in. 
#/groups/members/invite_do/40		ALL_OK		| -b cookie_p1 -c cookie_p1 -F "ids=3"

# Inviting a non-connected user to a private group that I am in. 
#/groups/members/invite_do/40		CTR_ERROR	| -b cookie_p1 -c cookie_p1 -F "ids=10"

# Inviting an hcp to a patient only private group that I am in
#/groups/members/invite_do/40		CTR_ERROR	| -b cookie_p1 -c cookie_p1 -F "ids=51"

# Inviting an patient to a hcp only private group that I am in
#/groups/members/invite_do/64		CTR_ERROR	| -b cookie_h51 -c cookie_h51 -F "ids=1"

# Inviting a user to a private group that i'm not in.
#/groups/members/invite_do/64		CTR_ERROR	| -b cookie_p1 -c cookie_p1 -F "ids=2"

########################################################################
\end{verbatim}

\begin{verbatim}
########################################################################
####################   TEST FILE FOR REFERRAL   ########################
########################################################################
#                                                                      #
# Use this syntax:                                                     #
#   <url>   <expected_result_string>   [ "|" <curl_extra_options> ]    #
#                                                                      #
# Examples:                                                            #
#   /home/sitemap    ALL_OK                                            #
#   /profile         ALL_OK      | -b cookie_file                      #
#                                                                      #
########################################################################

######################  LOG IN  ###########################

#LOG IN FOR PATIENT
/home/login REDIRECTED		| -F "email=patient1@email.com" -F "password=pat1" -c cookie_pat_1
/home/login	REDIRECTED		| -F "email=patient2@email.com" -F "password=pat2" -c cookie_pat_2

#LOG IN FOR DOCTOR
/home/login	REDIRECTED		| -F "email=doctor51@email.com" -F "password=doc51" -c cookie_doc_51
/home/login	REDIRECTED		| -F "email=doctor52@email.com" -F "password=doc52" -c cookie_doc_52
/home/login	REDIRECTED		| -F "email=doctor53@email.com" -F "password=doc53" -c cookie_doc_53
/home/login	REDIRECTED		| -F "email=doctor54@email.com" -F "password=doc54" -c cookie_doc_54
/home/login REDIRECTED		| -F "email=doctor55@email.com" -F "password=doc55" -c cookie_doc_55
/home/login	REDIRECTED		| -F "email=doctor56@email.com" -F "password=doc56" -c cookie_doc_56
/home/login	REDIRECTED		| -F "email=doctor57@email.com" -F "password=doc57" -c cookie_doc_57

#connect to hcp 56, they will is_referred_hcp
/connections/request/56      	              OK_MESSAGE        | -b cookie_doc_51
/connections/request/56      	              OK_MESSAGE        | -b cookie_doc_52
/connections/request/57      	              OK_MESSAGE        | -b cookie_doc_53 
/connections/request/57      	              OK_MESSAGE        | -b cookie_doc_54
/connections/request/57      	              OK_MESSAGE        | -b cookie_doc_55 

#accept connection request
/connections/accept/51		 	              ALL_OK	        | -b cookie_doc_56           
/connections/accept/52		 	              ALL_OK	        | -b cookie_doc_56
/connections/accept/53		 	              ALL_OK	        | -b cookie_doc_57
/connections/accept/54		 	              ALL_OK	        | -b cookie_doc_57
/connections/accept/55		 	              ALL_OK	        | -b cookie_doc_57

#set connection level between pat_id 2 and hcp_id 55 to high
/connections/change_level_do/55               OK_MESSAGE        | -F "level=3" -b cookie_pat_2


######################  CREATE REFERALS  #######################
#create two valid referrals for the same patient and the same hcp form different doctors
/refers/create_referral_do2/56           ALL_OK		       | -F "patient_id=1" -b cookie_doc_51
/refers/create_referral_do2/56           ALL_OK		       | -F "patient_id=1" -b cookie_doc_52
/refers/create_referral_do2/57           ALL_OK			   | -F "patient_id=2" -b cookie_doc_53
/refers/create_referral_do2/57           ALL_OK		       | -F "patient_id=2" -b cookie_doc_54

#high level doctor creating the referral
/refers/create_referral_do2/57           ALL_OK		       | -F "patient_id=2" -b cookie_doc_55

#create same referral again
/refers/create_referral_do2/56           CTR_ERROR         | -F "patient_id=1" -b cookie_doc_51

#create referral for patient not connected
/refers/create_referral_do2/56           CTR_ERROR         | -F "patient_id=45" -b cookie_doc_51

#create referral for hcp not connected
/refers/create_referral_do2/100          CTR_ERROR         | -F "patient_id=1" -b cookie_doc_51

#create referral for patient and hcp not connected
/refers/create_referral_do2/100          CTR_ERROR         | -F "patient_id=45" -b cookie_doc_51

#create referral for not valid patient id
/refers/create_referral_do2/56           CTR_ERROR         | -F "patient_id=123" -b cookie_doc_51

#create referral for not valid hcp id
/refers/create_referral_do2/123          CTR_ERROR         | -F "patient_id=1" -b cookie_doc_51

#patient creates a referral
/refers/create_referral_do2/51           CTR_ERROR         | -F "patient_id=2" -b cookie_pat_1


##############  ACCEPT REFERALS  ############################
#accept a referral
/refers/accept_referal/86                ALL_OK		      | -b cookie_pat_1

#accept a referal that has already been accepted
/refers/accept_referal/88                CTR_ERROR        | -b cookie_pat_2
/refers/accept_referal/89                CTR_ERROR        | -b cookie_pat_2
/refers/accept_referal/90                CTR_ERROR        | -b cookie_pat_2

#accept referral thats not mine
/refers/accept_referal/50                CTR_ERROR         | -b cookie_pat_1

#accpt referral that does not exist
/refers/accept_referal/200               CTR_ERROR         | -b cookie_pat_1

#hcp accepts the referral
/refers/accept_referal/86                CTR_ERROR         | -b cookie_doc_51
\end{verbatim}





\subsection{Test cases}

Each controller's functions were tested in several ways to anticipate possible user's actions. In general, they were tested for:
\begin{itemize}
\item \textbf{Value Types}
\begin{itemize}
\item Correctly typed values. This is to ensure that the functions do their basic purpose.
\item Incorrectly typed values. This is to prevent database errors or the SQL injection security threat.
\end{itemize}
\item \textbf{Authorization} In order to access a majority of the functionality, a user must be logged in. So, authorization was tested in the following ways: 
\begin{itemize}
\item  Valid login. By providing a valid login, we checked to see if the website capabilities were allowed. 
\item Invalid login. By providing an invalid login, we checked to see if the website capabilities were denied when the user tried to type the paths in the URL.
\item Not logged in. By providing not logging in, we checked to see if the website capabilities were denied when the user tried to type the paths in the URL.
\end{itemize}
\item\textbf{Accessibility} Per specifications, a health care provider and a patient have different permissions and functionality on the website. 
\begin{itemize}
\item Health Care Provider Accessibility. We tested that health care providers were only allowed to do permitted actions, per the specifications. For example, doctor's should not be able to request a patient connection. The action is never displayed as a link on the Salute website, so it would be difficult for a user to do this. However, we tested it by typing the path in the URL to assure that even if a user managed to find a way, the controller would not allow for such an action to be carried out. 
\item Patient Accessibility. We tested that patients were only allowed to do permitted actions, per the specifications. For example, a patient should not be able to request a connection with another patient. The action is never displayed as a link on the Salute website, so it would be difficult for a user to do this. However, we tested it by typing the path in the URL to assure that even if a user managed to find a way, the controller would not allow for such an action to be carried out. 
\end{itemize}
\item \textbf{Errors}
\begin{itemize} 
\item Query errors. If the controller calls a model function, and the model function experienced an error from the query, an error code is passed back to the controller function. The controller function then respectively prints an error code. 
\item Internal server errors. We included error catch code segments within the controller to handle the cases that weren't captured by the other tests. These cases would be where there was an internal server error. This helped for debugging our logic.  
\end{itemize}
\end{itemize}


\section{Database Tests}
The database was initially tested with simple SELECT * queries to see if the database tables were created and loaded correctly using the automated scripts\footnote{The bash script start\_everything.sh was used to automate the database creation process.  It used create\_tables.sql and load\_data.sql to create and populate the database.}. The database was populated with test data from thirteen different files (approximately one for every table)\footnote{These test files are: accounts.txt, appointments.txt, connections.txt, groups.txt, hcp\_account.txt, invite.txt, is\_in.txt, medical\_records.txt, messages.txt, patient\_account.txt, payment.txt, permissions.txt and refers.txt.}.  Then as the functions in each model were being implemented, the sql queries for each function would first be tested in the database to make sure that the correct result was returned.  Only after the sql query was tested and returned the correct information to the function in the model, would we determine that the function was fully implemented.  Additionally, error checking is done in each function to determine if an sql query sent to the database was successfully executed.

All of the test data files that are used to populate the database not only provide testing capabilities for the models, but also for the views and the controllers since they are interrelated.  For a list of all of the tests that the controllers test, read subsection \ref{subsec:controllertests}.

\subsection{Test Data} A script called generate\_data.sh was used to create all of the test data in most of the test data files\footnote{genetate\_data.sh does not generate test data for messages.txt, groups.txt, is\_in.txt, and invite.txt.  These were done manually.}.  Next we will describe all of the important test data files.

\subsection{accounts.txt} There are fifty patient accounts with account ids 1-50.  There are fifty hcp accounts with account ids 51-100.

\subsection{connections.txt} 
\paragraph{Patient to Patient Connections}The first twelve patient account ids are connected with four other patients.  Patient account id 13 is only connected with one patient, patient id 50.  No two patients are connected with the same patient.
\begin{verbatim}
example:
     pat_id  pat_id     pat_id  pat_id		
        1       2          2       6			
        1       3          2       7
        1       4          2       8
        1       5          2       9		
\end{verbatim}

\paragraph{Patient to HCP Connections} Each patient account is connected with five hcps.  Patient account id 50 is only connected with one hcp, hcp account id 100.  There is an overlap of four, which means two patients are connected with the same four hcps.
\begin{verbatim}
example:
     pat_id  hcp_id     pat_id  hcp_id		
        1       51         2       52
        1       52         2       53
        1       53         2       54
        1       54         2       55
        1       55         2       56
\end{verbatim}

\paragraph{HCP to HCP Connections}  Each hcp account is connected with two other hcps that are seven and eight account ids away respectively.  Hcp account id 93 is only connected with one hcp, hcp account id 100.  There is an overlap of one, which means two hcps are connected with the same hcp.
\begin{verbatim}
example:
     hcp_id  hcp_id     hcp_id  hcp_id		
        51      58         52      59
        51      59         52      60
\end{verbatim}

\subsection{appointments.txt}
\paragraph{Past Appointments}  Each patient has four past appointments with the first four hcps they are connected with.
\paragraph{Upcomming Appointments} Each patient has four upcomming appointments with the first four hcps they are connected with.  Two of them have been accepted by the hcp, two of them have not been accepted by the hcp.

\subsection{groups.txt}  There are six groups to represent all of the possible types of groups.  The possible types of groups are: public patient to patient, public patient to hcp, public hcp to hcp, private patient to patient, private patient to hcp, private hcp to hcp.

\subsection{invite.txt}
Sender account id, receiver account id and group id to represent all of the invitations to private groups that the sender belongs to.

\subsection{is\_in.txt}
Patient and hcp account ids, group id, level type to represent the groups that they belong to.

\subsection{medical\_records.txt}
Each patient has four medical records for their past four appointments.

\subsection{payment.txt}
Each patient has four bills that correspond to their past four appointments.  Two of them have been paid, two of them have not been paid.

\subsection{permissions.txt}
Each hcp has permission to view the medical records that they uploaded for each one of their past appointments.

\subsection{refers.txt}
Each hcp has two referrals for their fist connected patient.  Hcps stop making referrals at account id 93.  Hcp account id 93 has only one referral.  The patient gets referred to an hcp that is seven and eight account ids away. These are the hcps that the referring hcp is connected with.
\begin{verbatim}
example:
     refering_hcp_id     is_referred_hcp_id     pat_id	
             51                    58              1
             51                    59              1
             52                    59              2
             52                    60              2
\end{verbatim}

\chapter{Analysis}
All we did was an informal analysis, suck as code inspections of each other's code. We would like to perform some other, more formal, software analysis in the future.