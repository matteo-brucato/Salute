\section{About this document}

This document is aimed at providing a deep description of the Web-based application that we are developing for the class \emph{CS180 - Introduction to Software Engineering} at the University of California Riverside during Winter quarter 2011. Our purpose is to provide a documentation that would possibly allow new team members to join the project quickly and easily. At the same time this documentation will help us to keep an audit trail of what we did and how we did it. We will discuss all our researches, problems and decision.

The report is divided into 5 parts. The first one comprises this brief introduction along with an overview of the system we designed and implemented. The second part is a list of all the requirements and the current status of work, whereas the third part is a high-level view of the system, i.e its design and a low-level description, i.e. its implementation. Then, we added an operating manual for the end-user and an appendix with screen-shots and diagrams.

\subsection{Overview}
Our application is a web-site, running through https (secure http), to manage patient's healthcare documents and resources. The logic and behaviour is centered on patients, rather than on healthcare providers. In this view, all HCPs are considered as ''customers'' of patients. Patients can ask for ''social connections'' with HCPs(healthcare providers), can upload or download their medical records and can grant access to them to a specific set of healthcare providers. They can also both manage their appointments and bills.

\subsection{Description/Motivation}
We quote here the entire description given in the requirement paper for Milestone 0. We believe that this gives the best perspective in order to understand the whole system.
\begin{quotation}
   Modern medical offices rely on specialized software to manage the care and treatment of patients. Often,
such software “locks” data into a proprietary format and/or behind excessive security protocols. These
measures prevent the easy flow of information between patients and their doctors. As case in point, a
patients medical records are often seen as the “property” of the health care provider rather than the patients
personal property.
   
   The objective of this project is to create a web-based medical management portal in which a patient retains
control over their own medical records, yet provides easy access to any authorized healthcare provider. This
project is envisioned as part social network, part medical information database that will allow for patient
authorized “sharing” of relevant medical records to various doctors and specialists.
   
   This application will provide a web-based health information portal for (at least) two distinct groups
(patients and doctors). A patient will be allowed to create and manage their own profile, view and manage
their medical records, create and withdrawal “sharing links” to their selected health care providers, and
schedule/request appointments. Doctors will have a patient management service that will allow for the
scheduling/authorizing appointments, managing patient care, and billing functionality.
\end{quotation}


\section{Tools and Technology}
We immediately realized that the best way to implement a complex Web application like \emph{Salute} was using all the possible technologies that we were aware of. Firstly, we decided to use the well-known system design, \emph{MVC} (Model-View-Controller). We also decided to use a framework named \emph{CodeIgniter}\footnote{www.codeigniter.com} that is written in Php\footnote{www.php.net}. We fixed \emph{LAPP} as our system environment stack. LAPP stands for \emph{Linux}, \emph{Apache}, \emph{PostgreSQL}, \emph{Php}. Then we decided to use \emph{XHTML} as a mark-up language to describe web pages because it is standardized and is an XML that additionally uses a JavaScript framework called \emph{JQuery}\footnote{www.jquery.com}. \emph{JQuery} has a powerful set of tools that are easy to use and are browser-independent. In order to separate the concerns of ``what to display'' and ``how to display it'', we decided to use \emph{CSS} to manage layouts and presentational matters in the interface. We also decided to foresee the change and implement Ajax natively.

We chose \emph{Doxygen}\footnote{www.doxygen.com} as a parsable-formalized way to write comments in the code. Doing so, we were able to create a detailed documentation for the implementation, automatically. The results are a \emph{pdf} document and a \emph{html} easy-to-navigate website, both containing the same content. This makes for an easy reference of our code.

We also chose to use \emph{github}\footnote{www.github.com} as a version control for our project as our team worked together, and progressively added new features. This resource allowed to keep each developer up to date with the rest of the developer's progress and changes. \emph{Github} also provides a feature to ``issue bugs'', this allowed for each developer to post any bugs they encountered in their own implementations, or while testing others. \emph{Github's} version control made it easy to revert back to an older version if any feature broke. 

Last but not least, we decided to use \LaTeX as the language and tool for writing this documentation.

This is just a brief introduction to the tools and technologies that we used in our project. We will go into more details in the rest of this documentation.